\chapter{Bar Numbering}
The current bar number is stored in a count register call \keyindex{barno}.
When \verb|\startpiece| is encountered, \verb|\barno| is set equal to
another count register called \verb|\startbarno|, whose default value is one.
Therefore, if you want the first bar to have a number $n$ different from~$1$, you
may either say \verb|\startbarno=|$n$ before \verb|\startpiece|, or say
\verb|\barno=|$n$ afterwards, but before the first bar line. You may also alter the bar
number at any time, either by explicitly resetting \verb|\barno|, or by
incrementing it with a command like \verb|\advance\barno-1|.

\musixtex\ supports two distinct modes for printing bar numbers. In \ital{periodic}
bar numbering, the bar number is
placed above the top staff with a user-selectable frequency. In
\ital{system} bar numbering, the
number will appear at the beginning of each system.

\section{Periodic bar numbering}
In a normal piece, periodic bar number printing is turned on by default, with a frequency
of one. In an extract, the default is to not print bar numbers.
To turn off bar numbering say \keyindex{nobarnumbers}. To reinstate periodic
bar numbering, or to initiate it in an extract, say \keyindex{barnumbers}.
To change to a
different frequency $n$, say \verb|\def|\keyindex{freqbarno}\verb|{|$n$\verb|}|.

The appearance and positioning of the bar number is controlled by the
token \keyindex{writethebarno}, which by default is defined as\\
\verb|\def\writethebarno{\fontbarno\the\barno\kernm\qn@width}|
where the font is defined as \verb|\def\fontbarno{\it}|. You can change
either of these as desired, for example

  \medskip
\begin{music}\barnumbers
\parindent0pt\startpiece
\Notes\Dqbu gh\Dqbl jh\en
\notes\Dqbbu fg\Dqbbl hk\en\bar
\Notes\Tqbu ghi\Tqbl mmj\en
\def\fontbarno{\bf}%
\notes\Tqbbu fgj\Tqbbl njh\en\bar
\Notes\Qqbu ghjh\Qqbl jifh\en\bar
\notes\Qqbbu fgge\Qqbbl jhgi\en\endpiece
\end{music}
\noindent which was coded as
\begin{quote}\begin{verbatim}\barnumbers
\Notes\Dqbu gh\Dqbl jh\en
\notes\Dqbbu fg\Dqbbl hk\en\bar
\Notes\Tqbu ghi\Tqbl mmj\en
\def\fontbarno{\bf}%
\notes\Tqbbu fgj\Tqbbl njh\en\bar
\Notes\Qqbu ghjh\Qqbl jifh\en\bar
\notes\Qqbbu fgge\Qqbbl jhgi\en
\end{verbatim}\end{quote}

\section{System bar numbering}
To have a bar number printed just above the beginning of each system, use
\keyindex{systemnumbers}. The distance above the staff is controlled
by \verb|\raisebarno|, which by default is \verb|4\internote| (to fit
above a treble clef). This can be redefined with the command
\begin{quote}
\verb|\def|\keyindex{raisebarno}\verb|{|\ital{any \TeX~dimension}\verb|}|
\end{quote}
Similarly, the horizontal position is defined by \keyindex{shiftbarno}
which by default is \verb|0pt|.

The number normally is enclosed in a
box. If you don't like that, you may redefine the macro \verb|\writebarno|
which by default is defined as
\begin{quote}
\verb|\def|\keyindex{writebarno}\verb|{\boxit{\eightbf\the\barno\barnoadd}}|
\end{quote}
This uses the utility \musixtex\ macro \keyindex{boxit} which will enclose
any text string in a box.

Here are some possible alternate formats for system bar numbers:

\medskip
\begin{music}\nostartrule
\def\fontbarno{\it}%
\let\extractline\hbox
\startbarno=36
\hbox to \hsize{%
\hss
  \raise20pt\hbox{(a) }%
  \systemnumbers\startextract
    \Notes\wh g\en
  \zendextract
\hss
  \def\writebarno{\tenrm\the\barno\barnoadd}%
  \def\raisebarno{2\internote}%
  \def\shiftbarno{2.5\Interligne}%
  \raise20pt\hbox{(b) }%
  \systemnumbers\startextract
    \Notes\wh g\en
  \zendextract
\hss
  \def\writebarno{\llap{\tenbf\the\barno\barnoadd}}%
  \def\raisebarno{2\internote}%
  \def\shiftbarno{1.3\Interligne}%
  \raise20pt\hbox{(c) }%
  \systemnumbers\startextract
    \Notes\wh g\en
  \zendextract
\hss}
\end{music}
\noindent These were coded as
\begin{itemize}\setlength{\itemsep}{0ex}
 \item[(a)] (default)
 \item[(b)]
  \begin{verbatim}
\def\writebarno{\tenrm\the\barno\barnoadd}%
\def\raisebarno{2\internote}%
\def\shiftbarno{2.5\Interligne}%
  \end{verbatim}
 \item[(c)]
  \begin{verbatim}
\def\writebarno{\llap{\tenbf\the\barno\barnoadd}}%
\def\raisebarno{2\internote}%
\def\shiftbarno{1.3\Interligne}%
  \end{verbatim}
\end{itemize}

If the previous line does not stop with a bar rule, then the next
printed system bar number will immediately be followed by the contents
of the token \keyindex{writezbarno}, whose
default setting is the lower case character `\verb|a|'. You might want to
change this to `\verb|+|', in which case you should say
\verb|\def\writezbarno{+}|.

